\documentclass{article}

\usepackage[margin=1in]{geometry}

\title{Problem Set 1}
\author{Mark Xavier (xaviem01)}

\begin{document}
	\maketitle
	
	\begin{enumerate}
		\item Problem 1
		
			\begin{center}
				\begin{tabular}{|c|c|c|c|c|}
					\hline
					Task & T1 & T2 & T3 & T4 \\
					\hline
					Length & 12 & 42 & 48 & 54 \\
					\hline
				\end{tabular}
			
				\begin{tabular}{|c|c|c|}
					\hline
					Processor & P1 & P2 \\
					\hline
					Speed & 2 & 3 \\
					\hline
				\end{tabular}
			\end{center}
		
			\begin{enumerate}
				\item Characterize tree structures state space problem.
				
				\begin{itemize}
					\item The \textbf{states} are characterized as as either the empty state (no tasks assigned to the any processors) or a partially filled state (some $n$ tasks assigned to $m$ processors).  Technically the goal state (the state where all tasks are assigned some processor and the total time taken is less than deadline time $D$) is also a state. 
					
					\item The \textbf{operators} (or operations) are defined as actions that assign a given task to a processor, such that the time taken by that processor to finish the task plus the sum of times for all other tasks assigned to that processor do not take time $D$ or more to finish.
					
					In other words, assuming $p_i$ to be some processor in the set of processors $P$, and $t_j$ to be some task in the set of tasks $T$, and defining $t_j \in p_i$ as task $t_j$ being assigned to $p_i$, the operation of assigning $t_j \in p_i$ is only allowed if the following inequality is maintained:
					
					$$(\sum_{p \in P}\sum_{t \in p}t.length/p.speed) < D$$
					
					Where $D$ is the deadline time.
					
					\item The \textbf{branching factor} is 2, since at each step we assign a given task to one of the two processors.
					
					\item The \textbf{depth of the goal node} is known initially in some cases where a solution exists.  If a solution exists, then all tasks are assigned a processor, and therefore the depth of the goal node is 4, since there are 4 edges from the goal node to the root node.  Of course if no solution exists, then the goal node does not exist and it's depth is undefined.
				\end{itemize}
			
				\item State space with depth-first search:
				
				\begin{verbatim}
					O --- T1 to P1 (6,0) --- T2 to P1 (27,0) --- T3 to P1 (51 - Fail)
					                     |                   |
					                     |                   |-- T3 to P2 (27,16) --- T4 to P1 (54, 16 - Fail)
					                     |                                        |
					                     |                                        |-- T4 to P2 (27, 34 - Fail)
					                     |
					                     |-- T2 to P2(6,14) --- T3 to P1(30,14) --- T4 to P1(57,14 - Fail)
					                                                            |
					                                                            |-- T4 to P2(30, 32 - PASS)
				\end{verbatim}
			\end{enumerate}
		
	\end{enumerate}
\end{document}